\documentclass{jarticle}

\usepackage[ms]{pxchfon}% MSフォントを指定
\usepackage{twocolumn}

%\usepackage[dvi ps]{graphicx} %%画像を読み込む

\usepackage[dvipdfmx]{graphicx} %%画像を読み込む
\usepackage{here}%画像の配置場所の指定をプログラムの実行順に強制的に配置する
\newcommand{\setPicture}[1]{\includegraphics[width=0.9\linewidth]{/Users/watanabe_shouta/2024_saishuu/picture/#1}}
  

\usepackage{subfigure}
\usepackage{amsmath}          %%genfrac http://www.biwako.shiga-u.ac.jp/sensei/kumazawa/tex/form006.html
%\usepackage{newtxtext,newtxmath}
\usepackage{ulem}             %%http://biwako.shiga-u.ac.jp/sensei/kumazawa/tex/ulem.html     uline,uuline,uwave,sout,xoutなど
\usepackage{multirow}
\usepackage{chukan2020}       %%最後に読み込むこと!(最後に読み込まないと\textwidthなどの設定が反映されない)

\pagestyle{empty} %ページ番号を入れるときにはコメントアウトする

\begin{document}

\linesparpage{50}

\title{
柔軟かつ屈曲可能な胴体を有する魚型ロボットの開発
}
\etitle{
 Development of Fish-Type Robot with Flexible and Bendable Torso
}
\author{
研究者 渡部 翔太\;\;\;
指導教員 中西 大輔
}
\eauthor{
Keywords: Fish-Type Robot, Flexible and Bendable Torso
}

\maketitle

\thispagestyle{empty}  %1ページ目にページ番号を入れるときにはコメントアウトする

%%%%%%%%%%%%%%%%%%%%%%%%%%%%%%%%%%%%%%%%%%%%%%%%%%%%%%%%%%%%%%%%%%%%%%%%%%%%%%%
\section{緒言}
水中・水上の推進システムにはスクリュープロペラを用いた推進方法や,魚を模したロボットによる尾びれ推進などがあげられる\cite{ichi}.
中でも尾びれ推進は周辺環境へ影響を与えず,加速性・旋回性に優れている.こういった点から尾びれ推進を用いた魚型ロボットの開発は生態
系調査や災害への支援といった面において注目されている\cite{ni}.
これに対して我々は,様々な魚型ロボットを研究・開発してきた.その中で昨年度の研究では,胴体部分を屈曲可能なリンク構造にすることで体を
しならせる魚らしい動きの実現にも成功している.また,リンク間にできる隙間によって水をうまくかけていないという問題を解決するために,
柔軟な外皮をかぶせたロボット(図)を開発した\cite{san}.しかし,外皮がリンクにうまく追従せず,遊泳性能
が落ちてしまった.そこで本研究では外皮をリンクにうまく追従させ,かつ魚らしい動きを実現できるロボットの開発を目指す.また,外皮ありと
外皮なしで遊泳実験を行い,外皮による遊泳性能を検証する.
%%%%%%%%%%%%%%%%%%%%%%%%%%%%%%%%%%%%%%%%%%%%%%%%%%%%%%%%%%%%%%%%%%%%%%%%%%%%%%%

\section{リンク追従可能な柔軟外皮装着魚ロボット}
\subsection{魚型ロボットの開発}
図に開発した魚型ロボットを.魚型ロボットの構造を図に示す.
魚の泳ぎ方としてアジ型遊泳という代表的な遊泳方法があり,実際の魚に動作を近づけるべく,今回は3Dスキャンされたアジのデータをもとに,元の
サイズを2倍したサイズでロボットのボディを作成した.

ロボットは頭部,胴体部,外皮の3つからなり,頭部と胴体部をそれぞれ別の外皮で包む構造になっている.ただし,頭部内部は空気で満たし,胴体部は
外皮の内部に水をいれ、浸水させた状態にしている.
頭部には制御回路とバッテリを格納している.頭部内部を空気で満たし防水を行う方法としては,光造形方式の3Dプリンタで制作した頭部の上からシリコン
製の外皮をかぶせ,首元を締め付けることによって実現している.マイコンはM5stampPicoを使用しており、マイコンをwifiのアクセスポイ
ントとして機能させることでスマホを用いて遠隔で角度指定等が可能である.
胴体部に関して,胴体部の前半分は頭部と一体になっており、その部分に防水仕様(IP67)のサーボモータ(Flash Hobby, M45CHW)を配置している.
また,サーボモータには後述する動作のためにプーリを取り付けており、そこに糸を巻き付けている.
胴体部の後ろ半分は厚さ0.75mmの弾性樹脂板(ポリプロピレン板)とリンクで構成されており,胴体内部を浸水させるためにリンクには穴を開けている.
また,糸を通すために浸水用の穴とは別に2mm程度の穴も開けている.
胴体部の一番後ろには厚さ0.3mmのポリスチレン製の尾びれをつけており,遊泳軌道追従のためのトラッキングマーカーを取り付けている.
また,遊泳時に水中姿勢を水平に保つために、頭部,サーボ上部,弾性体などに重り25gと浮きをつけバランス調整を行った.

今回駆動方法として,魚のように緩やかに体を湾曲させることが可能なワイヤ駆動方式を採用している.
動作としてはサーボモータに取り付けられたプーリが左右に動き,ワイヤが巻き取られることによって,弾性体に取り付けられたリンクが引っ
張られ,弾性体をたわませる.それによって体をしならせ、水をかくことで遊泳する.

\section{柔軟外皮の作製}
今回,外皮を作製するため鋳造のように型にシリコンを流し入れることによって外皮の作成を行った.図に外皮作製に用いた型と中子を示す.中子とは鋳造に
おいて中空部を作るために使われているもので,型の間にはめ込んで使用する.今回は胴体部と頭部の皮をそれぞれ作成した.
胴体部の皮はリンクの動きに追従させるために,リンクがはまるような溝を外皮内部に作製することにした.また,外皮を曲げた時にできるしわを改善するために
多少ロボットのサイズよりも小さく外皮を設計し,しっかり張った状態で装着することにした.

\begin{figure}[t]
   \centering
   \setPicture{skinless.jpg}
   \vspace*{-4mm}
   \caption{柔軟外皮を有するロボット(皮なし)}
   \label{fig:bandable-Torso}
\end{figure}
\begin{figure}[t]
   \centering
   \setPicture{withskin.jpg}
   \vspace*{-4mm}
   \caption{柔軟外皮を有するロボット(皮あり)}
   \label{fig:structure}
\end{figure}

%図の配置にHを使うとコードの順に無理やりねじ込もうとするので見出しと本文の間隔が引き延ばされる可能性アリ.注意すべし
%noindentを使えばHで図を配置した後に起きる強制的な改行を抑制することができる

%%%%%%%%%%%%%%%%%%%%%%%%%%%%%%%%%%%%%%%%%%%%%%%%%%%%%%%%%%%%%%%%%%%%%%%%%%%%%%%
\section{まとめと今後の予定}

本稿では昨年度開発されたロボットを参考に試作機を作成した.今後は防水対策を施した試作機を用いて遊泳実験を行
う.そして柔軟な外皮をロボットにかぶせ,完全防水可能かつ,遊泳性能を向上させた魚型ロボットの開発を目指す.
さらに最終的には柔軟な外皮によって生まれるしわを無くすために外皮表面に鱗を付け,さらなる遊泳性能の向上を目指す.


%%%%%%%%%%%%%%%%%%%%%%%%%%%%%%%%%%%%%%%%%%%%%%%%%%%%%%%%%%%%%%%%%%%%%%%%%%%%%%%

% \bibliography{mylib}
% \bibliographystyle{junsrt}


\begin{thebibliography}{99}

   \bibitem{ichi}
   平田宏一, 春海一佳, 瀧本忠教, 田村兼吉, 牧野雅彦, 児玉良明, 冨田宏. 魚ロボットに関する基礎的研究. 海上技術安全研究所報告, Vol. 2, No. 3, pp. 281-307, 2003.

   \bibitem{ni}
   高田洋吾, 中西志允, 荒木良介, 脇坂知行. Piv 測定と3 次元数値解析による小型魚ロボット周りの水の流動状態と推進能力の検討(機械力学, 計測, 自動制御). 日本機械学会論文集C 編, Vol. 76, No. 763, pp. 665–672, 2010.

   \bibitem{san}
   中西大輔, 石原康平, 柔軟外皮を有する飛び移り座屈駆動式魚型ロボットの開発. ロボティクス・メカトロニクス講演会2024, 2P2-B10, 2024.

\end{thebibliography}
\end{document}
