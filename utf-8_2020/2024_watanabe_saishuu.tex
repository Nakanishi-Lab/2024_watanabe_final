\documentclass{jarticle}

\usepackage[ms]{pxchfon}% MSフォントを指定
\usepackage{twocolumn}


%\usepackage[dvi ps]{graphicx} %%画像を読み込む
\setlength\floatsep{0pt}
\usepackage[dvipdfmx]{graphicx} %%画像を読み込む
\usepackage{here}%画像の配置場所の指定をプログラムの実行順に強制的に配置する
\newcommand{\setPicture}[1]{\includegraphics[width=1\linewidth]{/Users/watanabe_shouta/2024_saishuu/picture/#1}}
  
\usepackage{subcaption}
\usepackage{amsmath}          %%genfrac http://www.biwako.shiga-u.ac.jp/sensei/kumazawa/tex/form006.html
%\usepackage{newtxtext,newtxmath}
\usepackage{ulem}             %%http://biwako.shiga-u.ac.jp/sensei/kumazawa/tex/ulem.html     uline,uuline,uwave,sout,xoutなど
\usepackage{multirow}
\usepackage{chukan2020}       %%最後に読み込むこと!(最後に読み込まないと\textwidthなどの設定が反映されない)


\pagestyle{empty} %ページ番号を入れるときにはコメントアウトする

\begin{document}

\linesparpage{50}

\title{
柔軟かつ屈曲可能な胴体を有する魚型ロボットの開発
}
\etitle{
 Development of Fish-Type Robot with Flexible and Bendable Torso
}
\author{
研究者 渡部 翔太\;\;\;
指導教員 中西 大輔
}
\eauthor{
Keywords: Fish-Type Robot, Flexible and Bendable Torso
}

\maketitle

\thispagestyle{empty}  %1ページ目にページ番号を入れるときにはコメントアウトする

%%%%%%%%%%%%%%%%%%%%%%%%%%%%%%%%%%%%%%%%%%%%%%%%%%%%%%%%%%%%%%%%%%%%%%%%%%%%%%%
\section{緒言}
水中・水上の推進システムにはスクリュープロペラを用いた推進方法や,魚を模したロボットによる尾びれ推進などがあげられる\cite{ichi}.
中でも尾びれ推進は周辺環境へ影響を与えず,加速性・旋回性に優れている.こういった点から尾びれ推進を用いた魚型ロボットの開発は生態
系調査や災害への支援といった面において注目されている\cite{ni}.
これに対して我々は,様々な魚型ロボットを研究・開発してきた.その中で昨年度の研究では,胴体部分を屈曲可能なリンク構造にすることで体を
しならせる魚らしい動きの実現にも成功している.また,リンク間にできる隙間によって水をうまくかけていないという問題を解決するために,
柔軟な外皮をかぶせたロボット(図)を開発した\cite{san}.しかし,外皮がリンクにうまく追従せず,遊泳性能
が落ちてしまった.そこで本研究では外皮をリンクにうまく追従させ,かつ魚らしい動きを実現できるロボットの開発を目指す.また,外皮ありと
外皮なしで遊泳実験を行い,外皮による遊泳性能を検証する.
%%%%%%%%%%%%%%%%%%%%%%%%%%%%%%%%%%%%%%%%%%%%%%%%%%%%%%%%%%%%%%%%%%%%%%%%%%%%%%%

\section{リンク追従可能な柔軟外皮装着魚ロボット}
\subsection{魚型ロボットの開発}
図\ref{fig:skin}に開発した魚型ロボットを.魚型ロボットの構造を図\ref{fig:structure}に示す.
魚の泳ぎ方としてアジ型遊泳という代表的な遊泳方法があり,実際の魚に動作を近づけるべく,今回は3Dスキャンされたアジのデータをもとに,元の
サイズを2倍したサイズでロボットのボディを作成した.

ロボットは頭部,胴体部,外皮の3つからなり,頭部と胴体部をそれぞれ別の外皮で包む構造になっている.ただし,頭部内部は空気で満たし,胴体部は
外皮の内部に水をいれ、浸水させた状態にしている.
頭部には制御回路とバッテリを格納している.頭部内部を空気で満たし防水を行う方法としては,光造形方式の3Dプリンタで制作した頭部の上からシリコン
製の外皮をかぶせ,首元を締め付けることによって実現している.また,頭部にワンタッチロック機能を搭載しており,バッテリやマイコンの取り出しが容易に行える.
マイコンはM5stampPicoを搭載しており、マイコンをwifiのアクセスポイントとして機能させることでスマホを用いて遠隔で角度指定等が可能である.
胴体部に関して,胴体部の前半分は頭部と一体になっており、その部分に防水仕様(IP67)のサーボモータ(Flash Hobby, M45CHW)を配置している.
また,サーボモータには後述する動作のためにプーリを取り付けており、そこに糸を巻き付けている.
胴体部の後ろ半分は厚さ0.75mmの弾性樹脂板(ポリプロピレン
\begin{figure}[H]
   \centering
   \setPicture{withskin.jpg}
   \caption{作製した魚ロボット}
   \label{fig:skin}
\end{figure}
\begin{figure}[H]
   \centering
   \setPicture{fish.pdf}
   \caption{魚ロボットの構造}
   \label{fig:structure}
\end{figure}
\begin{figure}[H]
 \begin{minipage}[t]{0.45\linewidth}
   \centering
   \setPicture{katata.jpg}
   \caption{型(胴体部)}
   \label{fig:kata}
 \end{minipage}
 \hspace*{0.05\columnwidth}
 \begin{minipage}[t]{0.45\linewidth}
   \centering
   \setPicture{kawa.jpg}
   \caption{完成した外皮}
   \label{fig:kawa}
 \end{minipage}
\end{figure}
\noindent 板)とリンクで構成されており,胴体内部を浸水させるためにリンクには穴を開けている.
また,糸を通すために浸水用の穴とは別に2mm程度の穴も開けている.
胴体部の一番後ろには厚さ0.3mmのポリスチレン製の尾びれをつけており,遊泳軌道追従のためのトラッキングマーカーを取り付けている.尾びれを固定するリンクには
プーリから伸びる糸の一端が取り付けられている.
また,遊泳時に水中姿勢を水平に保つために、頭部,サーボ上部,弾性体などに重り25gと浮きをつけバランス調整を行った.

今回駆動方法として,魚のように緩やかに体を湾曲させることが可能なワイヤ駆動方式を採用している.
動作としてはサーボモータに取り付けられたプーリが左右に動き,ワイヤが巻き取られることによって,弾性体に取り付けられたリンクが引っ
張られ,弾性体をたわませる.それによって体をしならせ、水をかくことで遊泳する.

\begin{figure*}[htbp]
  \centering
  \begin{tabular}{ccc}
   \begin{minipage}[c]{0.25\linewidth}
      \centering
      \setPicture{t1000.eps}
      \subcaption{t=1000msの場合}
      \label{fig:t1000}
   \end{minipage}
   \begin{minipage}[c]{0.25\linewidth}
      \centering
      \setPicture{t2000.eps}
      \subcaption{t=2000msの場合}
      \label{fig:t2000}
   \end{minipage}
   \begin{minipage}[c]{0.25\linewidth}
    \centering
    \setPicture{t3000.eps}
    \subcaption{t=3000msの場合}
    \label{fig:t3000}
   \end{minipage}
  \end{tabular}
  \caption{周期一定時の速度変化}
  \label{fig:period}
\end{figure*}

\begin{figure*}[htbp]
  \centering
  \begin{tabular}{ccc}
   \begin{minipage}[c]{0.25\linewidth}
      \centering
      \setPicture{an90.eps}
      \subcaption{an=90の場合}
      \label{fig:an90}
   \end{minipage}
   \begin{minipage}[c]{0.25\linewidth}
      \centering
      \setPicture{an135.eps}
      \subcaption{an=135の場合}
      \label{fig:an135}
   \end{minipage}
   \begin{minipage}[c]{0.25\linewidth}
    \centering
    \setPicture{an180.eps}
    \subcaption{an=180の場合}
    \label{fig:an180}
   \end{minipage}
  \end{tabular}
  \caption{振幅一定時の速度変化}
  \label{fig:amp}
\end{figure*}

\subsection{柔軟外皮の作製}
今回,外皮を作製するため鋳造のように型にシリコンを流し入れることによって外皮の作成を行った.図\ref{fig:kata}に外皮作製に用いた型と中子を示す.中子とは鋳造に
おいて中空部を作るために使われているもので,型の間にはめ込んで使用する.型と中子はロボットと同じようにスキャンデータからモデルを作成した.
今回は胴体部と頭部の皮をそれぞれ作成した.
胴体部の皮についてはリンクの動きに追従させるために,リンクがはまるような溝を外皮内部に作製することにした.また,外皮を曲げた時にできるしわを改善するために
設計したロボットの胴体モデルを90パーセント縮小させて外皮を設計し,装着した時にしっかり皮が張った状態になるようにした.

皮の作成手順は次のとおりである.
まず型と中子に離型剤(GSIクレオス 多目的離型剤)を塗布する.この工程を挟まないと,大きめのモデルを作成するときにシリコンが型に張り付いてしまい,取り出すときに
困難になる.次に型と中子をはめ合わせ,クランプでしっかり挟み込んで固定する.この隙間に工作用紙を挟み込み,シリコンが漏れ出すのを防ぐ.次にシリコン液(ECOFLEX30)
を用意し,真空ポンプを用いて真空脱泡を行う.この工程を行うことでシリコンに穴が開かないようにする.最後にシリンジを用いてシリコン液を型に流し込み,4時間以上放置
することでシリコンが固まり,外皮が完成する.

\section{遊泳実験}
\subsection{実験条件}
今回,外皮による遊泳性能の検証を行うためにロボットに外皮を装着した状態と未装着の状態とで直進遊泳実験を行い,速度を比較することにした.実験は2つ行い,
1つ目は弾性体を振る周期tが一定の状態で尾びれ振幅anを変化させたときの速度の変化について,2つ目は尾びれ振幅an一定の状態で周期tを変化させたときの速度の変化について実験を
行うことにした.1つ目の実験は周期tを1000ms~3000msで1000msごとに設定し,それぞれの周期で尾びれ振幅anを45~180度まで45度ずつ変化させて合計12個のパラメータについて
皮装着時,未装着時それぞれについて実験を行った.2つ目の実験は尾びれ振幅anを90~180度まで45度ごとに設定し,それぞれの振幅で周期tを600~1200msまで200msずつ変化させて
合計12個のパラメータについて皮装着時,未装着時それぞれについて実験を行った.

\subsection{実験結果}
図\ref{fig:period},図\ref{fig:amp}に実験結果を示す.今回,直進遊泳をさせたかったがやや時計回りに進路が傾くことがあった.実験結果から皮あり,皮なし両
方に共通して周期が短く,振幅が大きいと遊泳速度が向上するということが分かった.また,ほぼすべての実験において皮を装着した状態の魚ロボットの方が未装着時よりも速度が速くな
っていることもわかる.また,皮あり皮なしで実験結果に大きく差が出るのはan=180でtを600ms~1200msに変化させたときだった.

\section{考察}
まず,直進遊泳を行えなかった原因として,糸にかかるテンションに少し偏りがあったと考える.本来であれば左右の糸を同じ強さで引っ張る必要があるが,引っ張る力を測定して糸を張っていた
わけではないので少なからず偏りがあったと考える.今回の場合だと,右側の張力が左側よりも強くなっていたために時計回りに進路が傾いてしまったと考える.

次に,皮を装着した時としなかった時で速度に差が出たことについて考察する.今回外皮を装着したロボットの方が遊泳速度が速くなった要因としては,水を掻く部分の面積が外皮未装着の状態
の時よりも大きくなっていることにあると考える.外皮なしの状態だと水を掻けるのは弾性体と尾びれのみになり,その面積は1.09





%図の配置にHを使うとコードの順に無理やりねじ込もうとするので見出しと本文の間隔が引き延ばされる可能性アリ.注意すべし
%noindentを使えばHで図を配置した後に起きる強制的な改行を抑制することができる
%tで配置したend figureの後ろにvspaceを置くと文章が変に詰められる可能性があるので注意
%vspaceやhspaceは余白調整に便利なコマンドだが,なぜか本文の間隔が不自然に空いてしまうことがあるのでなるべく使わない(現在setcaptionで余白調整できないか検討中)

%%%%%%%%%%%%%%%%%%%%%%%%%%%%%%%%%%%%%%%%%%%%%%%%%%%%%%%%%%%%%%%%%%%%%%%%%%%%%%%
\section{まとめと今後の予定}

本稿では昨年度開発されたロボットを参考に試作機を作成した.今後は防水対策を施した試作機を用いて遊泳実験を行
う.そして柔軟な外皮をロボットにかぶせ,完全防水可能かつ,遊泳性能を向上させた魚型ロボットの開発を目指す.
さらに最終的には柔軟な外皮によって生まれるしわを無くすために外皮表面に鱗を付け,さらなる遊泳性能の向上を目指す.


%%%%%%%%%%%%%%%%%%%%%%%%%%%%%%%%%%%%%%%%%%%%%%%%%%%%%%%%%%%%%%%%%%%%%%%%%%%%%%%

% \bibliography{mylib}
% \bibliographystyle{junsrt}


\begin{thebibliography}{99}

   \bibitem{ichi}
   平田宏一, 春海一佳, 瀧本忠教, 田村兼吉, 牧野雅彦, 児玉良明, 冨田宏. 魚ロボットに関する基礎的研究. 海上技術安全研究所報告, Vol. 2, No. 3, pp. 281-307, 2003.

   \bibitem{ni}
   高田洋吾, 中西志允, 荒木良介, 脇坂知行. Piv 測定と3 次元数値解析による小型魚ロボット周りの水の流動状態と推進能力の検討(機械力学, 計測, 自動制御). 日本機械学会論文集C 編, Vol. 76, No. 763, pp. 665–672, 2010.

   \bibitem{san}
   中西大輔, 石原康平, 柔軟外皮を有する飛び移り座屈駆動式魚型ロボットの開発. ロボティクス・メカトロニクス講演会2024, 2P2-B10, 2024.

\end{thebibliography}
\end{document}
